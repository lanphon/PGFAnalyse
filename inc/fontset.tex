%\usepackage{xeCJK}%xelatex的中文支持,如果不添加此包中文断句和缩进将无法正常
\usepackage[slantfont,boldfont]{xeCJK}%允许斜体和粗体
\usepackage{fontspec}%字体设置相关
%\usepackage[T1]{fontspec}%字体设置相关

%\setmainfont[Mapping=tex-text]{Times New Roman}%英文默认字体,Serif(Roman)
%% 即rmfamily
%\setCJKmainfont{宋体}%中文默认字体,中文相对应的有宋体,仿宋,楷体和魏书
%%\setsansfont{Arial}%英文无衬线字体,即sffamily,Sans Serif(Sans,没有,来自于法语)
%\setsansfont{Verdana}%英文无衬线字体,即sffamily,Sans Serif(Sans,没有,来自于法语)
%% 常见的无衬线字体有Tahoma,Verdana,Arial
%\setCJKsansfont{黑体}%中文无衬线字体,即类似于黑体的效果
%\setmonofont{Monaco}%英文等宽字体,即ttfamily,Monospace(Typewriter)
%\setCJKmonofont{微软雅黑}%中文等宽字体

\setmainfont[Mapping=tex-text]{Times New Roman}%英文默认字体,Serif(Roman)
% 即rmfamily
\setCJKmainfont[ItalicFont={Adobe Heiti Std}, BoldFont={Adobe Kaiti Std}]{Adobe Song Std}
%中文默认字体,中文相对应的有宋体,仿宋,楷体和魏书

\setsansfont{Arial}%英文无衬线字体,即sffamily,Sans Serif(Sans,没有,来自于法语)
% 常见的无衬线字体有Tahoma,Verdana,Arial
\setCJKsansfont[ItalicFont={Adobe Kaiti Std}, BoldFont={Adobe Fangsong Std}]{Adobe Heiti Std}
%中文无衬线字体,即类似于黑体的效果

\setmonofont{Monaco}%英文等宽字体,即ttfamily,Monospace(Typewriter)
\setCJKmonofont{Adobe Thai}%中文等宽字体


% 定义西文字体
\newfontfamily\TimesNewRoman{Times New Roman}
\newfontfamily\Blackadder{Blackadder ITC}%Blackadder ITC字体
\newfontfamily\Brush{Brush Script MT}%
\newfontfamily\Edwardian{Edwardian Script ITC}%
%\newfontfamily\Freebooter{Freebooter Script}
\newfontfamily\Freestyle{Freestyle Script}%
\newfontfamily\French{French Script MT}
\newfontfamily\Kunstler{Kunstler Script}%斜体风格,适合作为签名
\newfontfamily\Vladimir{Vladimir Script}%斜体风格,很漂亮
% 定义中文字体
\setCJKfamilyfont{song}{SimSun}%宋体
\setCJKfamilyfont{hei}{SimHei}%黑体
\setCJKfamilyfont{kai}{KaiTi}%楷体
\setCJKfamilyfont{fang}{FangSong}%仿宋
\setCJKfamilyfont{xingkai}{STXingkai}%华文行楷

\newcommand{\songti}{\CJKfamily{song}}%定义别名
\newcommand{\heiti}{\CJKfamily{hei}}
\newcommand{\kaiti}{\CJKfamily{kai}}
\newcommand{\fangsong}{\CJKfamily{fang}}
\newcommand{\xingkai}{\CJKfamily{xingkai}}
% 定义各部分的字体
\newcommand{\footnotefont}{\familydefault \footnotesize}     % 脚注字体
%\newcommand{\footnotefont}{\songti\normalfont}     % 脚注字体
% 定义字体大小,以符合中文习惯
\newcommand{\wuhao}{\fontsize{10.5bp}{12.6pt plus .3pt minus .2pt}\selectfont}
