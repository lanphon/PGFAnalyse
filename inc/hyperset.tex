% Bookmark  (Chinese bookmark supported)
\usepackage[xetex,
%dvipdfmx,
CJKbookmarks,
pdfstartview=FitH, 
bookmarksnumbered=true,
bookmarksopen=true,
colorlinks=true,%注释掉此项则交叉引用为彩色边框
pdfborder=001,%注释掉此项则交叉引用为彩色边框
linkcolor=black,
citecolor=gray
]{hyperref}

%PDF元数据设置,属于hyperref宏包的内容,所有可选的内容在hyperrefmanual.pdf的第八
%页
%可以使用Adobe Acrobat查看这些元数据
\hypersetup{
pdftitle = {\modeltitlecn},%pdf标题
pdfsubject = {\modeltitlecn \modelversion},%主题
pdfkeywords = {\modelkeyword},%关键字
pdfauthor = {\modelauthornet .<\modelmailedu>},%作者
pdfcreator = {XeLaTeX with Hyperref Package},%不能使用latex的logo
pdfproducer = {lanphon@bupt.edu.cn},
%pdfpagemode=FullScreen%打开就是全屏,适用于beamer展示文档
}
