
\usepackage{tikz}
% We need lots of libraries...
\usetikzlibrary{%
  arrows,%
  calc,%
  fit,%
  patterns,%
  plotmarks,%
  shapes.geometric,%包含菱形等形状,可用于绘制流程图中判断节点
  shapes.misc,%
  shapes.symbols,%
  shapes.arrows,%
  shapes.callouts,%
  shapes.multipart,%
  shapes.gates.logic.US,%
  shapes.gates.logic.IEC,%
  er,%
  automata,%
  backgrounds,%
  chains,%
  topaths,%
  trees,%
  petri,%
  mindmap,%
  matrix,%
  calendar,%
  folding,%
  fadings,%
  through,%
  positioning,%
  scopes,%
  decorations.fractals,%
  decorations.shapes,%
  decorations.text,%
  decorations.pathmorphing,%
  decorations.pathreplacing,%
  decorations.footprints,%
  decorations.markings,%
  shadows}
%最后一项必须直接跟},不能换行,否则会出错

%我自己的流程图style,包括startstop(开始与结束框),passprocess(中间处理框
%),decision(判断框)以及line(线条)四种元素。
\tikzset{
passprocess/.style={
shade,
rectangle,
draw=red!50!black!40,
%minimum width=50pt,
%minimum height=20pt,
minimum size=13pt,
text width=2cm,%设置节点文本宽度达到自动换行的目的
text badly centered,
inner sep=2pt,%内容与边框之间的间距
top color=white,
bottom color=red!50!black!40,
font=\ttfamily\scriptsize
},
startstop/.style={
rectangle,%矩形
rounded corners=5pt,%圆角半径5pt
%minimum width=50pt,
%minimum height=20pt,
text centered,
%为什么无法达到填充的效果??
text width=2cm,%设置节点文本宽度达到自动换行的目的
draw=red!50!black!20,
top color=white,
bottom color=red!30!black!50,
%fill=orange,
%draw=red,
font=\ttfamily\scriptsize
},
decision/.style={
diamond,%形状,diamond为菱形
shape aspect=2,%更扁平的菱形,aspect之设定菱形宽度和长度的比值
%draw=green,%边框颜色
draw=red!50!black!20,
top color=white,
bottom color=red!70!black!20,
minimum size=13pt,
%fill=lime,%设定填充颜色
font=\ttfamily\scriptsize,%设定字体
text centered%文字居中
},%刚刚忘写了这个逗号,tex直接崩溃
line/.style = {
draw,
->,
%shorten>=2pt,
thick
}
}
